% \documentclass[CEJM,DVI]{cej} % use DVI command to enable LaTeX driver
\documentclass[ITESM,PDF]{cej} % use PDF command to enable PDFLaTeX driver
\usepackage{layout}

\begin{document}
\begin{titlepage}

  \newcommand{\HRule}{\rule{\linewidth}{0.5mm}}
  % Defines a new command for the horizontal lines, change thickness here

  \center % Center everything on the page

  %HEADING SECTIONS

  \textsc{\LARGE University Name}\\[1.5cm] % Name of your university/college
  \textsc{\Large Major Heading}\\[0.5cm] % Major heading such as course name
  \textsc{\large Minor Heading}\\[0.5cm] % Minor heading such as course title

  %TITLE SECTION
  
  \HRule \\[0.4cm]
         { \huge \bfseries Title}\\[0.4cm] % Title of your document
         \HRule \\[1.5cm]

         %AUTHOR SECTION
         
         \begin{minipage}{0.4\textwidth}
           \begin{flushleft} \large
             \emph{Author:}\\
             John \textsc{Smith} % Your name
           \end{flushleft}
         \end{minipage}
         ~ 
         \begin{minipage}{0.4\textwidth}
           \begin{flushright} \large
             \emph{Supervisor:} \\
             Dr. James \textsc{Smith} % Supervisor's Name
           \end{flushright}
         \end{minipage}\\[4cm]
         
         %DATE SECTION
         {\large\today}\\[30mm]
         % Date, change the \today to a set date if you want to be precise

         %LOGO SECTION
         %\includegraphics{Logo}\\[1cm]
         % Include a department/university logo - this will require the graphicx package


         \vfill % Fill the rest of the page with whitespace

\end{titlepage}

%\maketitle
%\baselinestretch{2}
\section{Introduction }

Some introductory text. Some introductory text. Some introductory text. Some introductory text. Some introductory text. Some introductory text. Some introductory text. Some introductory text. Some introductory text. Some introductory text. Some introductory text. Some introductory text. Some introductory text. Some introductory text. Some introductory text. Some introductory text. Some introductory text. Some introductory text. Some introductory text. Some introductory text. Some introductory text. Some introductory text. Some introductory text. Some introductory text. Some introductory text. Some introductory text.




\section{Sample section}


Sample theorem with citation:

\begin{theorem}[Pythagoras \cite{euclid00}]\label{Pythagoras-famous}
Let $a,b,c$ denote the sides of a triangle. If the angle between $a,b$ equals $90^\circ$ then
$$|a|^2 + |b|^2 = |c|^2.$$
Moreover, regardless of the assumption on the angle, it holds that $|a|+|b|\ge |c|$.
\end{theorem}

See References for the bibliography style in CEJM.
Below is a proposition with a proof.

\begin{proposition}\label{Standard-stuff}
The only possible real solutions of the quadratic equation $ax^2+bx+c=0$ are
$$x_1 = \frac{-b-\sqrt{\Delta}}{2a} \qquad\text{and}\qquad x_2 = \frac{-b+\sqrt{\Delta}}{2a},$$
where $\Delta = b^2-4ac$.
\end{proposition}

\begin{proof}
Observe that
\begin{equation}
(x-x_1)\,(x-x_2) = -{\frac{\Delta}{4\,a^2}}+x^2+{\frac{b\,x}{a}}+{\frac{b^2}{4\,a^2}}
\label{eq-somelabel}
\end{equation}
Taking into account that $\Delta = b^2-4ac$, equation (\ref{eq-somelabel}) leads to
\begin{equation}
(x-x_1)\,(x-x_2) = \frac{c}{a}+x^2 + \frac{b\,x}{a}
\label{eq-anotherlabel}
\end{equation}
Finally, (\ref{eq-anotherlabel}) shows that $a\,(x-x_1)\,(x-x_2)$ equals the original equation.
A polynomial of degree two cannot have more than two roots, therefore $x_1,x_2$ are the only possible solutions of our equation.
\end{proof}

\begin{corollary}
If $b^2<4ac$ then the quadratic equation $ax^2+bx+c=0$ has no real solutions.
\end{corollary}

\subsection{Subsections and sample formulas}

This is a subsection.

Typical aligned list of formulas with labels, displayed by using the \verb align environment.
\begin{align}
\label{eq:first}
\left(b+a\right)\,\left(2\,b-3\,a\right) &= 2\,b^2-a\,b-3\,a^2\\
\label{eq:second}
123+321 &= 444
\end{align}
The same list without labels.
\begin{align*}
\left(b+a\right)\,\left(2\,b-3\,a\right) &= 2\,b^2-a\,b-3\,a^2\\
123+321&=444
\end{align*}
Now the same list with custom tags.
\begin{align*}
\label{eq:smallstars}
\left(b+a\right)\,\left(2\,b-3\,a\right) &= 2\,b^2-a\,b-3\,a^2
\tag*{$(**)$}\\ %<-- \tag* does not add parentheses
\label{eq:doubledagger}
123+321&=444
\tag{$\ddagger$} %<-- \tag adds parentheses
\end{align*}
Another example of aligned formulas
\begin{align*}
  1+1&=2    &    1+2&=3    &    1+3&=4   \\
\intertext{with some text inserted between,}
 10+1&=11   &   10+2&=12   &   10+3&=13  \\
100+1&=101  &  100+2&=102  &  100+3&=103
\end{align*}
by using the \verb \intertext command.





\subsubsection{Polynomials of degree three}

This is a ``subsubsection".

Below is a complicated formula, which must be divided into several lines using, for instance, the
\verb align* \ environment:
\begin{align*}
x_1 &= \left(-{{\sqrt{3}\,i}\over{2}}-{{1}\over{2}}\right)\, \left({{3^ {- {{3}\over{2}} }\,\sqrt{12\,a^4-12\,a^3+188\,a^2-432\,a +247}}\over{2}}-{{2\,a^3+72\,a-81}\over{54}}\right)^{{{1}\over{3}}}\\
&+ {{\left({{\sqrt{3}\,i}\over{2}}-{{1}\over{2}}\right)\,\left(a^2-3 \right)}\over{9\,\left({{3^ {- {{3}\over{2}} }\,\sqrt{12\,a^4-12\,a^3 +188\,a^2-432\,a+247}}\over{2}}-{{2\,a^3+72\,a-81}\over{54}}\right) ^{{{1}\over{3}}}}}-{{a}\over{3}},\\
x_2 &= \left({{\sqrt{3}\,i}\over{2}}- {{1}\over{2}}\right)\,\left({{3^ {- {{3}\over{2}} }\,\sqrt{12\,a^4-12 \,a^3+188\,a^2-432\,a+247}}\over{2}}-{{2\,a^3+72\,a-81}\over{54}} \right)^{{{1}\over{3}}}\\
&+{{\left(-{{\sqrt{3}\,i}\over{2}}-{{1}\over{2 }}\right)\,\left(a^2-3\right)}\over{9\,\left({{3^ {- {{3}\over{2}} } \,\sqrt{12\,a^4-12\,a^3+188\,a^2-432\,a+247}}\over{2}}-{{2\,a^3+72\, a-81}\over{54}}\right)^{{{1}\over{3}}}}}-{{a}\over{3}},\\
x_3 &= \left({{3 ^ {- {{3}\over{2}} }\,\sqrt{12\,a^4-12\,a^3+188\,a^2-432\,a+247} }\over{2}}-{{2\,a^3+72\,a-81}\over{54}}\right)^{{{1}\over{3}}}\\
&+{{a^2 -3}\over{9\,\left({{3^ {- {{3}\over{2}} }\,\sqrt{12\,a^4-12\,a^3+188 \,a^2-432\,a+247}}\over{2}}-{{2\,a^3+72\,a-81}\over{54}}\right)^{{{1 }\over{3}}}}}-{{a}\over{3}}
\end{align*}

\begin{theorem}\label{th_poly3}
The numbers $x_1,x_2,x_3$ defined above are the only complex solutions of the equation
$$x^3+a\,x^2+x+3\,\left(a-1\right)=0.$$
\end{theorem}

\begin{proof}
Easy calculations, using \href{http://maxima.sourceforge.net/}{Maxima} or some other free computer algebra system.
\end{proof}


\paragraph{Other polynomials}

This is a paragraph with title.

One should admit that there is no general formula for the solutions of general polynomial equations and therefore sometimes we must use numerical methods.

\begin{example}\label{examp-123}
Consider the polynomial $v(t)=t^3+t+1$ of degree three. It turns out that its only real root is
$$t=\left({{3^ {- {{3 }\over{2}} }\,\sqrt{31}}\over{2}}-{{1}\over{2}}\right)^{{{1}\over{3 }}}-{{1}\over{3\,\left({{3^ {- {{3}\over{2}} }\,\sqrt{31}}\over{2}}- {{1}\over{2}}\right)^{{{1}\over{3}}}}}$$
There are also other two complex roots\footnote{Since the degree of $v$ is $3$, we know that there may be at most $3$ roots.}, which we ignore.
\end{example}


\begin{remark}\label{rem_CAS}
The proof of Theorem \ref{th_poly3} uses some advanced technology, including a computer algebra system. The same applies to Example \ref{examp-123}. On the other hand, Proposition \ref{Standard-stuff} is well known and its proof is rather elementary.
\end{remark}





\section{Tables and figures}

Table \ref{tab-01} shows how to show some data using the \verb table environment.

\begin{table}
\def~{\phantom0}
\catcode`\@=13
\def@{\phantom.}
\caption{Some caption text.\label{tab-01}}
\medskip
\begin{center}
\begin{tabular}{l|ccc}\hline
\multicolumn1l{\it Some title}\\\hline\hline
row 1, column 1         &   row 1, column 2  \\
row 2, column 1    &   row 2, column 2   \\
row 3, column 1    &   row 3, column 2   \\
\hline
\multicolumn1l{\it Another title} & Value 1 &   Value 2 &   Value 3 \\
\hline\hline
row 1                    & ~130 & 30 & 30 \\
row 2                    & 1025 & ~1 & 15 \\
row 3                   & ~100 & ~1 & 10 \\
row 4        & 2925 & ~1 & ~4 \\
row 5            & 2950 & ~1 & ~2 \\\hline
\end{tabular}
\end{center}
\end{table}

Figure \ref{fig_abc} shows how to use the \verb figure environment for displaying graphics, etc.

\begin{figure}
\caption{The graph of $y=x\,sin x$ in the interval [-50,50], created by wxMaxima 0.7.4. \label{fig_abc}}
%\vskip3cm
\includegraphics{xsinx-49.png}
\end{figure}


\section*{Acknowledgements}

The author(s) would like to thank some institutions for support and so on.


\begin{thebibliography}{9}

\bibitem{annis-etal} Anninis K., Crabi T.J., Sunday T.J., New methods for parallel computing, In: Lyonvenson S. (Ed.), Proceedings of Computer Science Conference (1-�10 Jul. 2007 Haifa Israel), University Press, 2007, 13�-179

\bibitem{p1} Author N., Coauthor M., Title of article, J. Some Math., 2007, 56, 243--256

\bibitem{katish} Katish A., The inconsistency of ZFC, preprint available at
http://arxiv.org/abs/1234.1234

\bibitem{kittel_etal} Kittel S.J., Maria G., Tuke M., Sepran D.J., Smith J., Tadeuszewicz K., et al., New class of measurable functions, J. Real Anal., 1997, 999, 234-�255

\bibitem{nowak1} Nowak P., New axioms for planar geometry, Eastern J. Math., 1999, 1, 324�-334, (in Polish)

\bibitem{nowak} Nowak P., Even better axioms for planar geometry, Eastern J. Math., (in press, in Polish), DOI: 33.1122/321

\bibitem{euclid00} Pythagoras S., On the squares of sides of certain triangles, J. Ancient Math., 2003, 4, 1--30, (in Greek)

\bibitem{sambrook} Sambrook J., Uncountable abelian groups, In: Sambrook J., Russell D.W. (Eds.), Contributions to Abelian groups, 3rd ed., Nauka, Moscow, 2001

\bibitem{sambrook-russell} Sambrook J., Russell D.W., Abelian groups, 3rd ed., Nauka, Moscow, 2001

\end{thebibliography}

\end{document}
